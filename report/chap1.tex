\chapter{Introduction}

\section{Introduction to GPU}
A \textbf{Graphics Processing Unit (GPU)}, also occasionally called \textbf{visual processing unit (VPU)}, is a specialized electronic circuit designed to rapidly manipulate and alter memory to accelerate the building of images in a frame buffer intended for output to a display. GPUs are used in embedded systems, mobile phones, personal computers, workstations, and game consoles. Modern GPUs are very efficient at manipulating computer graphics, and their highly parallel structure makes them more effective than general-purpose CPUs for algorithms where processing of large blocks of data is done in parallel. In a personal computer, a GPU can be present on a video card, or it can be on the motherboard or—in certain CPUs—on the CPU die. \\\\

\section{Introduction to CUDA}

\textbf{CUDA} also known as \textbf{Compute Unified Device Architecture} is a parallel computing platform and programming model created by NVIDIA and implemented by the graphics processing units (GPUs) that they produce. CUDA gives developers access to the virtual instruction set and memory of the parallel computational elements in CUDA GPUs. Using CUDA, the latest Nvidia GPUs become accessible for computation like CPUs. Unlike CPUs, however, GPUs have a parallel throughput architecture that emphasizes executing many concurrent threads slowly, rather than executing a single thread very quickly. This approach of solving general-purpose (i.e., not exclusively graphics) problems on GPUs is known as GPGPU.

